\chapter{Conclusões e Trabalhos Futuros}

Com o objetivo de realizar um estudo de validação do ETHEL, \textit{software} desenvolvido por uma equipe de pesquisadores e programadores da UESC. É apresentada neste trabalho uma metodologia de validação baseada em um estudo comparativo. Com esta finalidade se analisou a utilização do \textit{software} \textit{Posturography Test}, por ser uma solução que também utiliza a WBB na implementação de de exames posturográficos, e por já ter sido validado (\citeauthor{llorens2016posturography}, \citeyear{llorens2016posturography}). Além disso, está disponível de forma gratuita para uso. A metodologia  de validação, consiste na realização de testes com um mesmo grupo de pessoas. Na efetivação dos testes, deve ser utilizada a mesma WBB como plataforma de força e os pacientes devem executar a mesma sequência de testes em ambos os \textit{softwares} e, em condições semelhantes. Com isso, se ao fim dos testes, os dados coletados com os \textit{softwares} apresentarem uma correlação aceitável, pode-se inferir que o ETHEL apresenta dados válidos. 

Para viabilizar o estudo comparativo, o presente estudo  analisou as métricas estimadas no \textit{Posturography Test}. Após análise, foi constado que a Excursão Máxima AP e a Excursão máxima ML, que são utilizadas no \textit{Posturography Test}, não estavam implementadas no ETHEL. Sendo assim passaram ser calculadas e foram incluídas no conjunto de métricas do mesmo. 

A interface gráfica do modulo LOS foi desenvolvida e está funcional. Além disso, já é possível realizar a captura do deslocamento do COP do participante, que desloca o seu centro de massa sobre a WBB na direção do alvo indicado. Entretanto, as métricas calculadas a partir do LOS. (tempo de reação, excursão máxima e controle direcional) ainda não são estimadas pelo ETHEL. Esta limitação se deve à falta de descrição clara para realização do cálculo das mesmas, principalmente no \textit{Posturography Test}, ferramenta utilizada na proposta de validação do ETHEL.

Na realização dos testes com humanos, apresentados neste trabalho, O ETHEL apresentou excelente funcionamento. O que sugere que após o termino da implementação do módulo LOS, ele já estará pronto para execução da proposta de validação descrita. Ademais, o grupo de pesquisadores que cederam os dados dos testes apresentados, relataram que o ETHEL mostrou uma redução significativa no tempo de execução do exames, quando comparado com os testes preliminares feitos com o \textit{Posturography Test}. O que pode se tornar um fator decisivo na escolha de utilização entre os dois \textit{softwares}. Além disso, o mesmo grupo de voluntários deveria ser submetido a avaliação semelhante utilizando o \textit{Posturography Test}, possibilitando assim, ter uma prévia da eficácia da metodologia de de validação proposta. Entretanto, por indisponibilidade no sistema WEB esta etapa da avaliação não foi possível realizar o estudo comparando as duas aplicações. 

Visando aprimorar o protocolo de execução de aquisição dos dados e a redução do tempo na realização dos exames, ao fim dos testes, foi calculado o desvio da primeira repetição de cada condição em relação a média das repetições da condição em questão, na forma de erro relativo. Conforme os resultados apresentados no \textbf{\textbf{Capitulo} \ref{resultados}}, a métrica velocidade média apresentou um desvio da primeira repetição em relação a média das repetições, menor que 10\%. Além do mais, apresentou um desvio padrão médio próximo a zero. Isso pode indicar a não necessidade de realização de três repetições, quando se desejar estimar a velocidade média do COP. Principalmente na condição sensorial olhos fechados sem espuma, que foi a condição sensorial em que se apresentou melhores resultados. Entretanto, não é possível inferir que não se faz necessária a realização de três repetições para cada condição na realização dos testes. Pois, o estudo em questão realizou os testes com uma amostra pequena de pacientes. 


Como possíveis trabalhos futuros pode-se destacar o desenvolvimento e a integração ao ETHEL dos cálculos das métricas no módulo LOS. Deve-se realizar, também, testes com uma amostra maior de pacientes, visando aprimorar o protocolo de execução de aquisição dos dados e verificar a necessidade de realização de repetições para cada condição no cumprimento dos testes. Finalmente deve ser  realizado o estudo comparativo aqui proposto, como o objetivo de validar os resultados obtidos com o ETHEL. 
  
%O presente estudo analisou as métricas estimadas no Posturography Test. Após análise, foi constado que a Excursão Máxima AP e a Excursão máxima ML eram métricas que estavam presentes no Posturography Test, mas não pertenciam ao ETHEL. Sendo assim passaram ser calculadas e foram incluídas no mesmo. 

%A interface gráfica do modulo LOS foi desenvolvida e está funcional. Além disso, já é possível realizar a captura do deslocamento do COP do participante, que desloca o seu centro de massa sobre a WBB na direção do alvo indicado. Entretanto, as medidas apresentadas como resultado (tempo de reação, excursão máxima e controle direcional) ainda não são estimadas pelo ETHEL, devido a falta de descrição clara para realização do cálculo das mesmas, principalmente no Posturography Test ferramenta utilizada na proposta de validação do ETHEL.

%Visando aprimorar o protocolo de execução de aquisição dos dados e a redução do tempo na realização dos exames. Ao fim dos testes, foi calculado o desvio da primeira repetição de cada condição em relação a média das repetições da condição em questão, na forma de erro relativo. Conforme os resultados apresentados na Tabela \ref{tabelaResultados}, 51,85\% dos exames apresentaram um desvio da primeira repetição com a média das repetições, menor que 10\%. No entando, não é possível inferir que não se faz necessária a realização de três repetição para cada condição na realização dos testes. Pois, o estudo em questão realizou os testes com uma amostra pequena de pacientes. E uma análise de distribuição de Frequência em Intervalo de Classe deve ser feita para cada uma das quatro condições, afim de detalhar a distribuição dos erros de acordo com as condições sensoriais. 

%A proposta de validação para o software ETHEL é a realização do estudo comparativo com o software Posturography Test. Ele foi escolhido por ser uma solução que também utiliza a WBB na realização de exames posturográficos e por já ter sido validado (\cite{llorens2016posturography}, \citeyear{llorens2016posturography}). Além disso, está disponível de forma gratuita. A metodologia  de validação, consiste na realização de testes com um mesmo grupo de pessoas. Na realização dos testes, as pessoas utilizaram a mesma WBB como plataforma de força e executaram a mesma sequência de testes em ambos os softwares e, em condições semelhantes. Com isso, se ao fim dos teste, os dados coletados com os softwares apresentarem uma correlação aceitável, podemos inferir que o ETHEL apresenta dados válidos.           

%Na realização dos testes com humanos, apresentados neste trabalho. O ETHEL apresentou excelente funcionamento. O que sugere que após o termino da implementação do módulo LOS, ele já estará pronto para execução da proposta de validação descrita. Ademais, o grupo de pesquisadores que cederam os dados dos testes apresentados, relataram que o ETHEL mostrou uma redução significativa no tempo de execução do exames, quando comparado com o Posturography Test. O que pode se tornar um fator decisivo na escolha de utilização entre os dois softwares.

%Possíveis trabalhos futuros são:
%\begin{enumerate}
%    \item Desenvolver e integrar os cálculos das métricas no módulo LOS;
%    \item Realizar testes com uma amostra maior de pacientes, visando aprimorar o protocolo de execução de aquisição dos dados e verificar a necessidade de realização de repetições para cada condição no cumprimento dos testes.
%    \item Efetuação da proposta de validação descrita neste trabalho.
%\end{enumerate}
  
