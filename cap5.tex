\chapter{Conclusões e Trabalhos Futuros}

A partir deste trabalho foi possível realizar o desenvolvimento de uma API que tenha a capacidade de extrair dados textuais de arquivos PDF. E em vista da existência de uma grande demanda por informações encontradas em tais arquivos, nota-se a importância desta API.

Apesar da complexidade interna destes arquivos, existem várias ferramentas que são capazes de auxiliar na leitura dos objetos que o compõe. E a utilização de uma ferramenta de auxilio a leitura do arquivo mostrou o caminho para uma abordagem mais simples e prática do que as apresentadas por \citeauthor{sasirekhatext}, \citeauthor{ajedig2011pdf}, \citeauthor{marinai2009metadata} e \citeauthor{lin2011mathematical}

Este trabalho também contou com um forte estudo de ferramentas atuais e relevantes como, por exemplo o Swagger e JWT, e assim foi demonstrado que a união destas pode resultar em um software conciso e confiável. Além disso, foi possível notar a importância de uma comunidade ativa de desenvolvedores. O que contribui para a evolução e discussão destas ferramentas, já que sua maior utilização vem de empresas.

Após a construção do sistema \textit{front-end} para a classificação de TCC's do curso de Ciência da Computação da UESC, foram testadas as capacidades da API. Sendo algumas delas, a autenticação de usuário, o registro de parâmetros, o envio de arquivos, o retorno das informações esperadas dado os parâmetros, e o acesso à informação extraída.
Com estes testes foi notado o bom funcionamento e desempenho do sistema, e como pode ser notado nas imagens apresentadas no capítulo anterior, as extrações foram bem sucedidas. 

O sistema de classificação de TCC's provou a fácil utilização da API e se mostrou apto para utilização do colegiado. Deve-se observar que é um sistema \textit{front-end} de baixa complexidade, porém aliado a API, com poucos recursos este se tornou uma poderosa ferramenta para extração de dados de TCC, e assim rapidamente consegue cumprir a demanda da rede de egressos.

A conclusão deste projeto demonstra que há possibilidades de diferentes abordagens para a extração de dados em documentos PDF. O desenvolvimento aqui realizado comprovou que uma aproximação mais modular pode ser muito benéfica e sendo bem executada resulta em um sistema preciso.

Ademais com a criação desta aplicação foi possível imaginar cenários de múltiplos benefícios através do uso da API, e quesitos em que ela pode ser melhorada. Possíveis trabalhos futuros são:
\begin{enumerate}
    \item Aprimorar o módulo para extração de dados tabulares, mantendo a organização original das informações;
    \item Desenvolver sistema intuitivo para criação de parâmetros de extração; Auxiliando na criação de RegEx e coletando as coordenadas de forma gráfica;
    \item Aceitar outros formatos de arquivo \item Realizar testes de integração e escalabilidade;
    \item Criar teste automatizados.
\end{enumerate}