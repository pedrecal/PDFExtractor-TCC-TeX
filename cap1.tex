\chapter{Introdução}
A posturografia é utilizada como importante técnica para avaliar o equilíbrio e as oscilações corporais (\citeauthor{prieto1996measures}, \citeyear{prieto1996measures}). A melhor forma para  implementação desta técnica é a posturografia computadorizada que utiliza plataformas de força (PF) . Estes  equipamentos podem fornecer dados quantitativos detalhados acerca do equilíbrio corporal (\citeauthor{hendrickson2014relationship}, \citeyear{hendrickson2014relationship}). As plataformas de forças  estimam a oscilação do centro de pressão (COP) do paciente em análise, e permitem avaliar as alterações em relação a uma amostra saudável correspondente (\citeauthor{llorens2016posturography}, \citeyear{llorens2016posturography}).  A grande limitação para a utilização da posturografia é o alto custo das plataformas de forças comerciais. No brasil pode se encontrar alguns modelos, como por exemplo, a plataforma de força FP-4060-08-2000 BERTEC, que custa US 29.812,00 (\citeauthor{CarciPF}). Outra limitação é a necessidade de um espaço dedicado na clínica para seu uso (\citeauthor{visser2008clinical}, \citeyear{visser2008clinical}).

O desenvolvimento de acessórios e dispositivos para jogos tem evoluído significativamente nos últimos anos. Um exemplo disso é a Plataforma Wii Balance Board (WBB), lançada em 2007 como um controle de jogo para o sistema Nitendo  Wii, desenvolvido com o objetivo de promover uma maior imersão do usuário ao jogo. A WBB possui componentes similares a uma plataforma de força tradicional, dispõe de quatro sensores de carga medidores de tensão, capazes de obter dados sobre movimentos do COP e comunicar-se via \textit{bluetooth} com um computador. Ela foi avaliada como uma alternativa ás plataformas de força de nível laboratorial, devido ao seu baixo custo e fácil manuseio (< 4 kg) (\citeauthor{clark2018reliability}, \citeyear{clark2018reliability}). Atualmente, no Brasil a WBB pode ser adquirida por cerca de 100 reais (\citeauthor{AmericanasWBB}). Diversos artigos recentes têm comprovado a viabilidade de utilizar a WBB na realização de exames posturográficos  (\citeauthor{clark2010validity}, \citeyear{clark2010validity}), (\citeauthor{young2011assessing},
\citeyear{young2011assessing}),  (\citeauthor{leach2014validating}, \citeyear{leach2014validating}) (\citeauthor{clark2018reliability}, \citeyear{clark2018reliability}). Os estudos mostram que é possível avaliar quantitativamente as oscilações corporais dos pacientes e identificar a contribuição de cada um dos sistemas somatossensoriais na manutenção do equilíbrio.

Não são muitas as soluções disponíveis, que permitem utilizar a WBB integrada a um sistema para implementar exames de posturografia. Na revisão sistemática feita por Clark et al. (\citeyear{clark2018reliability}), percebe-se que a maioria dos autores constroem rotinas utilizando o \textit{software LabVIEW} (\textit{National Instruments}, Texas, EUA) para aquisição e leitura de dados. A principal limitação do uso dessa solução, é que o \textit{software} \textit{LabVIEW} é um software proprietário e sua licença é dispendiosa. Atualmente, no brasil a versão base custa R\$ 1.299,00/ano (\citeauthor{NationalInstruments}). Por outro lado o Posturography Test, está disponível para uso de forma gratuita e sua validação já foi efetuada por Llorens et al. (\citeyear{llorens2015low}) e (\citeyear{llorens2016posturography}). Suas principais deficiências consistem na dependência de estabelecer conexão com a internet para seu funcionamento, e o fato de ser um \textit{software} de código fechado e pouco documentado.

Neste contexto, na Universidade Estadual de Santa Cruz (UESC), encontra-se em estágio avançado o projeto de desenvolvimento do software ETHEL para implementação de protocolos de posturografia utilizando a WBB.  Este programa está sendo utilizado em um projeto do mestrado em saúde da UESC, projeto este que tem aprovação do Comitê de Ética em Pesquisa (CEP) da UESC (\textbf{Anexo} \textbf{\ref{anexoCEP}})  para realização de testes com humanos. Parte dos dados obtidos por eles serão cedidos ao presente trabalho, possibilitando assim a elaboração de uma metodologia para validação do ETHEL. Este \textit{software} permite capturar e processar os sinais gerados pela WBB e calcular parâmetros quantitativos utilizados na posturografia. Os primeiros resultados deste projeto foram apresentados por (\citeauthor{Thales2018}, \citeyear{Thales2018})  e (\citeauthor{Bizerra2018}, \citeyear{Bizerra2018}). Com as funcionalidades implementadas no ETHEL é possível utilizar a plataforma de jogos como uma alternativa de baixo custo aos equipamentos atualmente disponíveis, para implementação de técnicas de instrumentalização de exames e procedimentos clínicos.

No entanto, se faz necessária a validação dos resultados obtidos. Em Leach et al., (\citeyear{leach2014validating}) se apresenta uma validação utilizando ao mesmo tempo a WBB e uma plataforma de força comercial, para medir simultaneamente o deslocamento unidimensional do COP. No experimento de validação a WBB foi colocada sobreposta à plataforma de força e acima delas foi colocado um sistema mecânico de pêndulo invertido que era responsável por simular oscilações posturais unidimensionais. Dadas as oscilações, realizava-se a aquisição dos dados gerados por ambas plataformas. Por fim foi feito o estudo comparativo dos dados adquiridos para realização da confirmação da validade.

Tendo em vista o difícil acesso e o alto custo das plataformas de força tradicionais, o presente trabalho propõe uma metodologia de validação para o \textit{software} ETHEL realizando um estudo comparativo com uma solução que esteja disponível, que também utilize a WBB para aplicação de exames posturográficos e que já tenha sido aceita e validada. Por este motivo, para realização deste experimento o \textit{software} escolhido foi \textit{Posturography Test}, desenvolvido pelo grupo. Para efetuar a validação será necessário implementar no ETHEL o conjunto de métricas geradas no. Além disso, para realização dos testes, deverá ser utilizada a mesma plataforma com os dois softwares, em condições semelhantes, com o mesmo grupo de pessoas.
Ao fim dos testes, espera-se responder, se a metodologia aplicada e as métricas utilizadas pelo ETHEL na aquisição dos dados são válidos e podem ser utilizada como alternativa na realização de exames posturográficos.

\section{Objetivos}

\subsection{ Objetivo Geral}
 Propor um protocolo de validação do \textit{software} ETHEL como ferramenta para realização de exames posturográficos, com base na utilização do \textit{software} \textit{Posturography Test}.

\subsection{Objetivos Específicos}
\begin{enumerate}
\item Analisar as métricas do \textit{software} \textit{Posturography Test};

\item Implementar as métricas do \textit{Posturography Test} no ETHEL;

\item Implementar no ETHEL o módulo de \textit{ Limits of Stability} (LOS);

\item Aprimorar os protocolos de aquisição de dados visando reduzir o tempo de realização do exame;

\item Idealizar um protocolo para implementar um estudo comparativo.

\end{enumerate}

\section{Organização do Trabalho}
Este trabalho é organizado com a seguinte estrutura: O capítulo 2 apresenta a revisão da literatura, demonstra todo o embasamento teórico utilizado na construção deste estudo. O capítulo 3 explica todos os materiais e métodos empregados para chegar ao resultado final. O capítulo 4 expõe os resultados obtidos com testes realizados no ETHEL e apresenta as novas métricas desenvolvidas. Por fim, o capítulo 5 conclui o trabalho e apresenta as propostas de trabalhos futuros.
