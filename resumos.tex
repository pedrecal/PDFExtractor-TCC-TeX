% ---
% RESUMOS
% ---

% resumo em português
\setlength{\absparsep}{18pt} % ajusta o espaçamento dos parágrafos do resumo
\begin{resumo}
Para permitir a aplicação de protocolos de posturografia utilizando a \textit{Wii Balance Board} (WBB; Nintendo, Kyoto, Japão), um grupo de pesquisadores da Universidade Estadual de Santa Cruz (UESC) são responsáveis pela produção do sistema ETHEL, que já conta com um protótipo em estagio avançado de desenvolvimento. Antes de passar para a etapa de testes clínicos com pacientes é necessário à validação dos resultados obtidos com o ETHEL. Sendo assim, o presente trabalho propõe %fazer um estudo 
uma metodologia para fazer um estudo comparativo dos softwares ETHEL e \textit{Posturography Test}, desenvolvido pelo grupo \textit{Neurorehabilitation \& Brain Research Group},  que já foi testado e validado como ferramenta para realização de testes de avaliação clínica para controle de postura e de equilíbrio, utilizando a WBB. %O estudo tem como objetivo validar os resultados obtidos com o ETHEL. 
Para realização do estudo comparativo foram aprimorados e desenvolvidos protocolos no ETHEL, com intuito de reproduzir as métricas utilizadas pelo \textit{Posturography Test}. Como resultado temos uma técnica que deve permitir corroborar a metodologia de calibração e obtenção dos dados no o ETHEL, assim como otimizar os protocolos de medição. Desta forma disponibilizamos um protótipo que pode ser utilizado para um estudo comparativo dos softwares ETHEL e \textit{Posturography Test}.
\vspace{\onelineskip}
 
\noindent 
Palavras-chave:  \textit{Wii Balance Board}, Posturografia, \textit{Posturography Test}, ETHEL.
\end{resumo}

% resumo em inglês
%\begin{resumo}[Abstract]
% \begin{otherlanguage*}{english}
 % Realizar a verificação ortográfica, sintaxe e semântica.

 %  \vspace{\onelineskip}
 
  % \noindent 
  % \textbf{Keywords}: 
 %\end{otherlanguage*}
%\end{resumo}

%% resumo em francês 
%\begin{resumo}[Résumé]
% \begin{otherlanguage*}{french}
%    Il s'agit d'un résumé en français.
% 
%   \textbf{Mots-clés}: latex. abntex. publication de textes.
% \end{otherlanguage*}
%\end{resumo}
%
%% resumo em espanhol
%\begin{resumo}[Resumen]
% \begin{otherlanguage*}{spanish}
%   Este es el resumen en español.
%  
%   \textbf{Palabras clave}: latex. abntex. publicación de textos.
% \end{otherlanguage*}
%\end{resumo}
% ---
